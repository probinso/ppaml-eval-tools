%first_line
\documentclass[11pt]{article} % letterpaper is american!

\usepackage[british,UKenglish,USenglish,english,american]{babel}
\usepackage[pdftex]{graphicx}
\usepackage{epstopdf}

\usepackage{amsfonts,amsmath,amsthm,amssymb}

%\usepackage[T1]{fontenc}

\usepackage{tikz,pgf}
\usetikzlibrary{fit}

\pagestyle{empty}
\setlength{\parindent}{0mm}
\usepackage[letterpaper, margin=1in]{geometry}
%\usepackage{showframe}

\usepackage{multicol}
\usepackage{enumerate}

%%%%%%%%%%%%%%%%%%%%%%%%%%%%%%%%%%%%%%%%%%%%%%%%%%%%%%%%%%%%%%%%%%%%%%%%%%
%
%
%
%%%%%%%%%%%%%%%%%%%%%%%%%%%%%%%%%%%%%%%%%%%%%%%%%%%%%%%%%%%%%%%%%%%%%%%%%%
\setlength\columnsep{30pt} 
% this is a hack needed to prevent \item counters from overlapping text

\raggedcolumns
\makeatletter
\newcount\branch@start% as \setcounter isn't global
\newcount\branch@end
\newenvironment{branch}[1][2]{\begin{multicols}{#1}%
  \branch@start\c@enumi% save value of enumi for subsequent leaves
  \branch@end\c@enumi%   need to keep track of largest enumi in all leaves
  }{%set enumi to largest value in the leaves and close multicols
  \ifnum\c@enumi<\branch@end%
    \global\c@enumi\branch@end% we NEED this to take effect globally
  \fi%
\end{multicols}}
\newcommand\newleaf{%
  \columnbreak%
  \ifnum\c@enumi>\branch@end%
    \global\branch@end\c@enumi%
  \fi%
  \setcounter{enumi}{\branch@start}%
}
\makeatother
%%%%%%%%%%%%%%%%%%%%%%%%%%%%%%%%%%%%%%%%%%%%%%%%%%%%%%%%%%%%%%%%%%%%%%%%%%

\usepackage{color}
\usepackage{verbatim}
\usepackage{listings}
\lstset{
  basicstyle=\scriptsize\ttfamily,
  backgroundcolor=\color{lightgray},
  language=bash,
  frame=L,
}


\usepackage{xspace}
\usepackage{url}
\usepackage{cite}

\usepackage{titlesec}
\titlespacing*{\subsubsection}{0pt}{*0}{*0}
\titlespacing*{\subsection}{0pt}{0pt}{*0}
\titlespacing*{\section}{0pt}{0pt}{*0}

\newcommand{\Bold}{\mathbf}


%\setlength{\parskip}{1em}
%\setlength{\parindent}{1em}


\title{Use cases for PPAML-Eval-Tools}

\date{\today}


\def\pmr{Philip Robinson\xspace}
\def\ms{Matt Stottile\xspace}
\def\cf{Chris Fahlbusch\xspace}

\newenvironment{slimlist}{
  \begin{itemize}
    \setlength{\topsep}{0pt}
    \setlength{\itemsep}{1pt}
    \setlength{\parsep}{0pt}
    \setlength{\parskip}{0pt}
}{\end{itemize}}

\newenvironment{mitemize}[1]{
  % designed to prevent labeled lists from being broken up over page or column boundries. 
  \begin{minipage}{0.9\linewidth}
  \subsubsection*{#1}
  \begin{slimlist}
}{\end{slimlist}\end{minipage}}


\begin{document}
\pagestyle{empty}
\clearpage\maketitle
  

\newcommand{\form}[1]{{\em #1}\xspace}
\def\ptk{\form{peval-toolkit}}
\def\cp{\form{challenge problem}}
\def\ev{\form{quantitative evaluator}}
\def\rep{\form{report}}
\def\ds{\form{dataset}}
\def\ins{\form{instance}}
\def\sol{\form{solution}}
\def\gen{\form{generator}}
\def\out{\form{output}}
\def\tm{\form{team}}
\def\rn{\form{run}}
\def\eng{\form{engine}}

\def\G{\form{Galois}}
\def\D{\form{DARPA}}

\begin{mitemize}{\cp}
\item add a \cp to \ptk
  \begin{quote}
    This is effectively a place holder, the purpose of a \cp is to allow us to index more abstractly into an \ins. A \cp should be stored as a description and/or site-reference. 
    \end{quote}
\item register an \ins to a \cp
  \begin{quote}
    A \sol will address a specific \ins. An \ins needs a description a human readable id$_\#$ a time-period index and an \ev. 
    
    %You should be able to print number of \rn{s} from each \tm for each \ins. You should also be able to ask for all instances from one time period. 
  \end{quote}
\item register a \ds to a \ins
  \begin{quote}
    A \ds should have fetch/\gen instructions (seed needed if produced by \gen). We should be able to verify a \ds's uniqueness and existence separately.
    
    register can also be used to update a \ds{s} directory link. If this is done correctly, we should be able to automatically register \ds{s} that are used (... this may not be true wrt description and retrieval ...)

    {\em\footnotesize also if we are using digests for naming, then we will likely want to have a blocked name as defined by digest of empty directory.}
  \end{quote}
\item unregister \ds, \ins
  \begin{quote}
    If anything is unregistered then we should remove corresponding \rn{s}. The reason to unregister anything is if it was incorrect and needs to be purged. (if we need more disk space we physically delete directory rather than unregister)
  \end{quote}
  
\item look-up a \cp's description and site reference
\item look-up a \ds's fetch/generator instructions
\item look-up which \ins and \ds status 
  \begin{quote}
    Print the \tm{s} that have provided a \sol (and number of \rn{s}) for each \ds in a time-period index or \ins .
    
    The return format should allow us to resolve gaps cleanly when possible. Also should be able to fairly skip missing \ds{s}.
  \end{quote}
\item unpack \ds for use by \ev or \sol if on disk at location

\item revise \ds
\begin{quote}
  we should not need to revise unless we allow for automatic registration and need to update look-up, this is likely a case that is so infrequent that it can be ignored. 
\end{quote}
\end{mitemize}

\def\t24{\form{TA2-4}}

\begin{mitemize}{\eng}
\item as \t24, I want to submit an \eng, so that it can be used by my \sol{s} in the evaluation process
  \begin{enumerate}
  \item \t24 \tm develops an \eng
  \item \t24 packages \eng for submission
    \begin{enumerate}
    \item \t24 needs packaging spec
      \begin{enumerate}
        
      \item tar.bz2 of complete \eng (not partial directory trees)
      \item git url and tag
      \end{enumerate}
    \item \G needs installation instructions
    \item \G needs configuration instructions
    \item \G needs a smoke test
      \begin{enumerate}
      \item minimal code to verify \eng installation and configuration
      \item smoketest.sh wrapping minimal code (any Unix shell)
      \item logs and errors report to stderr
      \item return code is 0 on success, non-zero on failure
      \end{enumerate}
    \end{enumerate}
  \item \t24 submits \eng to \G
    \begin{enumerate}
    \item git repository with tagged releases
    \item our secure site
    \end{enumerate}
  \item \G installs and configures \eng
  
    \begin{branch}
    \item \G successfully executes \t24's smoke test manually 
    \item \G adds \eng to \ptk's database 
    \item ppaml use pps.ini -- {\bf to be discussed later after understanding cps submission}
    \item \G informs \t24 team of success smoke test results
      \begin{enumerate}
      \item e-mail
      \end{enumerate}
      
      \newleaf

    \item \G cannot successfully execute \t24's smoke test
    \item \t24 receives failure notification from \G
      \begin{enumerate}
      \item e-mail
      \item crash log
      \end{enumerate}
    \end{branch}
  \item this completes this use case
  \end{enumerate}
\item as \G, I want to run an \eng against multiple \sol{s}, so that I can use a \ev
  %\item as \G, I want to know which \eng{s} have been submitted within a time-period index, so that I can quickly check on \tm{s}' participation status
  %\item as \t24 and \G, I want to run the \ev, so that I can generate \rep{s} and see my performance status
\item[backlog::] extend running multiple \eng{s} against a \sol to \tm{s}
\end{mitemize}

\def\de{\form{domain expert}}
\begin{mitemize}{\cp}
\item as a \de, I have been selected to provide a \cp, so that it can be used in the evaluation process
  \begin{enumerate}
    
  \item as a \de, I want to 
    \begin{enumerate}
    \item 
      
    \item \ds for testing, training, and evaluation
    \item (optional) dataset  \gen
    \item \ev description
    \item \ev implementation
    \item reference implemenation
      
    \end{enumerate}
  \item 
  \end{enumerate}
\end{mitemize}
  % change enumeration to be engine.value
\end{document}
