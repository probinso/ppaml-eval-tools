\documentclass[8pt,letterpaper]{article} % letterpaper is american!
\usepackage[letterpaper, margin=.75in]{geometry}
%\usepackage{showframe}
\usepackage{multicol}
\usepackage{enumerate}

\usepackage[british,UKenglish,USenglish,english,american]{babel}
\usepackage[pdftex]{graphicx}
\usepackage{tikz,pgf,epstopdf}
%\usepackage{amsfonts,amsmath,amsthm,amssymb}

\usetikzlibrary{fit}

\pagestyle{empty}
\setlength{\parindent}{0mm}

%%%%%%%%%%%%%%%%%%%%%%%%%%%%%%%%%%%%%%%%%%%%%%%%%%%%%%%%%%%%%%%%%%%%%%%%%%
%
%
%
%%%%%%%%%%%%%%%%%%%%%%%%%%%%%%%%%%%%%%%%%%%%%%%%%%%%%%%%%%%%%%%%%%%%%%%%%%
\setlength\columnsep{30pt} 
% this is a hack needed to prevent \item counters from overlapping text
\raggedcolumns
\makeatletter
\newcount\branch@start% as \setcounter isn't global
\newcount\branch@end
\newenvironment{branch}[1][2]{\begin{multicols}{#1}%
  \branch@start\c@enumi% save value of enumi for subsequent leaves
  \branch@end\c@enumi%   need to keep track of largest enumi in all leaves
  }{%set enumi to largest value in the leaves and close multicols
  \ifnum\c@enumi<\branch@end%
    \global\c@enumi\branch@end% we NEED this to take effect globally
  \fi%
\end{multicols}}
\newcommand\newleaf{%
  \columnbreak%
  \ifnum\c@enumi>\branch@end%
    \global\branch@end\c@enumi%
  \fi%
  \setcounter{enumi}{\branch@start}%
}
\makeatother
%%%%%%%%%%%%%%%%%%%%%%%%%%%%%%%%%%%%%%%%%%%%%%%%%%%%%%%%%%%%%%%%%%%%%%%%%%

\usepackage{color}
\usepackage{verbatim}
\usepackage{listings}
\lstset{
  basicstyle=\scriptsize\ttfamily,
  backgroundcolor=\color{lightgray},
  language=bash,
  frame=L,
}

\usepackage{xspace}
\usepackage{url}
\usepackage{cite}

\usepackage{titlesec}
\titlespacing*{\subsubsection}{0pt}{*0}{*0}
\titlespacing*{\subsection}{0pt}{0pt}{*0}
\titlespacing*{\section}{0pt}{0pt}{*0}

\newcommand{\Bold}{\mathbf}

\title{Use cases for PPAML-Eval-Tools}
\date{\today}

\def\pmr{Philip Robinson\xspace}
\def\ms{Matt Stottile\xspace}
\def\cf{Chris Fahlbusch\xspace}

\newenvironment{slimlist}{
  \begin{itemize}
    \setlength{\topsep}{0pt}
    \setlength{\itemsep}{1pt}
    \setlength{\parsep}{0pt}
    \setlength{\parskip}{0pt}
}{\end{itemize}}

\newenvironment{mitemize}[1]{
  % designed to prevent labeled lists from being broken up over page or column boundries. 
  %\begin{minipage}{\linewidth}
  \subsection*{#1}
  \begin{slimlist}
}{
\end{slimlist}
%\end{minipage}
\vspace{1em}
\pagebreak
}

\begin{document}
\pagestyle{empty}
\clearpage%\maketitle

\newcommand{\form}[1]{{\em #1}\xspace}
\def\ptk{\form{peval-toolkit}}
\def\cp{\form{challenge problem}}
\def\tev{\form{quantitative evaluator}}
\def\lev{\form{qualitative evaluator}}
\def\rep{\form{report}}
\def\ds{\form{data-set}}
\def\gs{\form{ground-truth}}
\def\ins{\form{instance}}
\def\sol{\form{solution}}
\def\gen{\form{generator}}
\def\out{\form{output}}
\def\tm{\form{team}}
\def\rn{\form{run}}
\def\eng{\form{engine}}
\def\t24{\form{TA2-4}}
\def\de{\form{domain expert}}
\def\desc{\form{INI File}}

\def\G{\form{Galois}}
\def\D{\form{DARPA}}
\def\U{\form{UNKNOWN}}



\begin{mitemize}{\cp}
\item add a \cp to \ptk
  \begin{quote}
    This is effectively a place holder, the purpose of a \cp is to allow us to index more abstractly into an \ins. A \cp should be stored as a description and/or site-reference. 
    \end{quote}
\item register an \ins to a \cp
  \begin{quote}
    A \sol will address a specific \ins. An \ins needs a description a human readable id$_\#$ a time-period index and an \tev. 
    
    %You should be able to print number of \rn{s} from each \tm for each \ins. You should also be able to ask for all instances from one time period. 
  \end{quote}
\item register a \ds to a \ins
  \begin{quote}
    A \ds should have fetch/\gen instructions (seed needed if produced by \gen). We should be able to verify a \ds's uniqueness and existence separately.
    
    register can also be used to update a \ds{s} directory link. If this is done correctly, we should be able to automatically register \ds{s} that are used (... this may not be true wrt description and retrieval ...)

    {\em\footnotesize also if we are using digests for naming, then we will likely want to have a blocked name as defined by digest of empty directory.}
  \end{quote}
\item unregister \ds, \ins
  \begin{quote}
    If anything is unregistered then we should remove corresponding \rn{s}. The reason to unregister anything is if it was incorrect and needs to be purged. (if we need more disk space we physically delete directory rather than unregister)
  \end{quote}
  
\item look-up a \cp's description and site reference
\item look-up a \ds's fetch/generator instructions
\item look-up which \ins and \ds status 
  \begin{quote}
    Print the \tm{s} that have provided a \sol (and number of \rn{s}) for each \ds in a time-period index or \ins .
    
    The return format should allow us to resolve gaps cleanly when possible. Also should be able to fairly skip missing \ds{s}.
  \end{quote}
\item unpack \ds for use by \tev or \sol if on disk at location

\item revise \ds
\begin{quote}
  we should not need to revise unless we allow for automatic registration and need to update look-up, this is likely a case that is so infrequent that it can be ignored. 
\end{quote}
\end{mitemize}


\end{document}
